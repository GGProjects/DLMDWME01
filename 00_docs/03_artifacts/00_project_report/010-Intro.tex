\chapter{Einleitung}

Im gewählten Szenario für diese Fallstudie geht es um die Herausforderung, die Planungslogik des Bereitschaftsdienstplans für den Berliner Rotkreuz-Rettungsdienst zu verbessern. Ursprünglich aus dem Modul Model Engineering stammend \citep{pak_aufgabenstellung_dlmdwme01_2024}, wurde diese Aufgabenstellung, nach Rücksprache mit den jeweiligen Tutoren, für zwei weitere Module adaptiert und konnte so von der Use Case Analyse bis zur Umsetzung in thematisch aneinander gereihten Modulen bearbeitet werden. 

\begin{enumerate}
  \itemsep-8pt
  \item \textbf{Data Science UseCase:} Analyse des Anwendungsfalls bis zur Präsentation für die Freigabe durch das Management.
  \item \textbf{Technische Projektplanung:} Aufbauend auf dem PitchDeck vom Projektstrukturplan über eine Kostenschätzung und das Risikomanagement bis zur Stakeholder-Analyse.
  \item \textbf{Model Engineering:} Die Umsetzung des UseCase als Projekt in einem Git-Repository mit bereitgestellten Daten bis zu einem fertiggestellten Vorhersagemodel.
\end{enumerate}

Diese Arbeit behandelt den letzten Teil der Umsetzung dieses Anwendungsfalles. 

Die Bearbeitung der Problemstellung aus mehreren Blickwinkeln hilft, ein durchgängiges Verständnis für die Abwicklung realer Szenarien zu erlangen. Die ganzheitliche Betrachtungsweise der Thematik lässt einen tieferen Einblick in die Prozesse der Data Science entstehen, erhöht die Identifikation mit dieser Aufgabenstellung und macht somit schlussendlich auch mehr Spaß. Da diese Arbeit den letzten der drei Abschnitte umfasst, kann an dieser Stelle bereits festgehalten werden, dass der Lerneffekt durch diese Vorgehensweise und Kombination mehrerer Module definitiv profitiert hat. Der Teufel steckt oft im Detail und, in dem Wissen, dass man,  aufgrund der begleitenden Aufgabenstellung, von anfänglichen Fehlern und Unsauberkeiten wieder zu einem späteren Zeitpunkt eingeholt wird, beschäftigt man sich von Beginn an, intensiver mit dem Thema und versucht eben auch die Details zu lösen.

Bei dieser Aufgabe handelt es sich um ein Regressionsproblem in Zeitreihendaten und damit um die Vorhersage eines numerischen Wertes aufgrund historisch strukturierter Daten. Gelöst wurde die Aufgabe schlussendlich durch die lineare Abhängigkeit der Zielvariablen von einem modifizierten weiteren Merkmal, das besser durch saisonale Zeitreihenmuster zu beschreiben war.

	
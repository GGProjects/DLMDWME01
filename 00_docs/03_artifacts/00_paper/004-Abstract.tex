\chapter{Abstract}

Der Berliner Rot Kreuz Rettungsdienst (DRK Berlin) projektiert den Einsatz eines Machine Learning Systems zur effizienten Planung eines Bereitschaftsdienstplanes für Einsatzfahrer:innen. Auf Basis einer erfolgten Use Case Analyse, stellt diese Arbeit den Projektplan vor und behandelt sowohl die Projektstruktur und -teamplanung als auch die Zeitplanung, die Kostenschätzung, die Stakeholderanalyse und das Risikomanagement des Projektes.

Für den Projektmanagement-Ansatz wird eine hybride Vorgehensweise verfolgt. Agile Methoden, wie Scrum, werden mit klassischen Elementen des V-Modell XT kombiniert. Das schafft eine klare Projektstruktur mit definierten Phasenübergängen und Planungswerkzeugen und gewährleistet gleichzeitig die Vorteile der flexiblen und iterativen Entwicklung. Das V-Modell XT wurde auf das notwendigste beschränkt um ein kleines Projektteam, einen geringen \glqq Overhead\grqq\ und eine modulare Struktur zu ermöglichen.

Mit den verwendeten Elementen des V-Modell XT wurde in manchen Bereichen zwar über die Vorgaben der Aufgabenstellung hinausgeschossen, die Dokumentation des Rahmenprozesses bietet jedoch eine hervorragende Unterstützung bei gleichzeitiger Ermutigung zur Flexibilität. Man bekommt das Gefühl, für den Projektstart \glqq bereit\grqq\ zu sein. Der, in der Aufgabenstellung verlangte, Projektstrukturplan kommt in der Dokumentation des V-Modell XT \citep{angermeier_v-modell-xt_2024} nicht vor. Leider ist auch in den vorgestellten Modellen des Studienmoduls \glqq Management von IT-Projekten\grqq\ \citep{iu_internationale_hochschule_management_2024}, das diesem Modul üblicherweise vorangeht, kein Projektstrukturplan abgebildet. Den ersten zufriedenstellenden Anhalt dazu lieferte \cite{wikipedia_projektstrukturplan_2022}. Beim Projektplan finden die Anforderungen der Aufgabenstellung sowie das V-Modell XT aber wieder zusammen. Als mögliche Ansatzpunkte wurden weiters The PM Minimalist Quick Start Guide \citep{greer_pm_2011} sowie die Lektüre von How Big Things Get Done \citep{flyvbjerg_how_2023} herangezogen, diese lieferten jedoch nicht die gehoffte Unterstützung für die Aufgabe.

Die inhaltliche Vorgabe des Projektes ist es, kosteneffizient die tägliche Anzahl des benötigten Bereitschaftspersonals vorherzusagen, gleichzeitig jedoch nie zu wenig Einsatzfahrende als Reserve vorzusehen. Das System und das Projekt bauen sich rund um diese Vorgabe auf. Die Planung sieht vor mit einem Entwicklungsteam in mehreren, teils parallelen Sprints, diese Minimalanforderung umzusetzen, sinnvoll um weitere Funktionen zu erweitern und das System, mit entsprechender Hardware und benutzerfreundlichen Frontends, in eine stabile Umgebung einzubetten, die sich gut in den IT-Betrieb des DRK Berlin integrieren lässt. Die wahrscheinlich langen Lieferzeiten der Hardwarekomponenten geben der Entwicklung einen zeitlichen Rahmen für die Umsetzung der Sprints. Inhaltlich wird dieses Projekt nicht als \glqq zeitkritisch\grqq\ eingestuft. Gegebenenfalls wirkt sich daher eine Verzögerung zwar auf die, der Kostenstelle des Projektes zugeschriebenen, Personalkosten jedoch nicht auf sonstige Aspekte und Abhängigkeiten der Umsetzung aus. Der Fokus der Entwicklung liegt, auch in der Priorisierung der Arbeitspakete, auf den, für das Machine Learning Modell relevanten Aufgaben. Das Vorhersagemodell selbst bildet das Herzstück der Aufgabe. In der Zeit- und Kostenschätzung wurde daher der Einsatz eines/r weiteren Data Scientist zwar erwogen, jedoch auch wieder verworfen. 

Anhand der exemplarisch dargestellten Risiken, ist auch ersichtlich, dass nicht nur ein potentieller Reputationsschaden für das DRK (aufgrund einer zu niedrig prognostizierten Anzahl nötiger Bereitschaftsfahrender und damit mangelnder Einsatzbereitschaft) das Projekt gefährden kann, sondern auch eine mögliche Ablehnung durch die Nutzer, die durch die Umsetzung mehr Verantwortung übernehmen müssen als in den derzeitigen Arbeitsprozessen. Dies sind u.a. zwei wesentliche Punkte, denen im Rahmen der Projektumsetzung, bei der Gestaltung des Modells und der Regelung der zukünftigen Arbeitsprozesse, unbedingt Rechnung getragen werden muss. Die zukünfigen Nutzer werden daher bereits ab Projektbeginn sowohl für das Requirements Engineering als auch für die Qualitätssicherung hinzugezogen. 




\section{Making of}

Ausgangspunkt dieser Arbeit war ein MachineLearning Szenario, das dem Modul \glqq Model Engineering\grqq\ \citep{pak_aufgabenstellung_dlmdwme01_2024} dieses Studienganges entnommen wurde. Dank der Zustimmung der zuständigen Tutoren wurde es möglich, dasselbe Thema in drei unabhängigen aber inhaltlich dennoch aneinander anschließenden Modulen zu bearbeiten. Dadurch ergab sich eine durchgängige Bearbeitung des Problemstellung durch mehrere Entstehungsphasen eines Machine Learning Systems hindurch. 

\begin{enumerate}
  \itemsep-8pt
  \item \textbf{Modul \glqq Data Science UseCase\grqq\ :} Analyse des Anwendungsfalls (ML-Canvas) bis zur Präsentation für die Freigabe durch das Präsidium.
  \item \textbf{Modul \glqq Technische Projektplanung\grqq\ :} die gegenständliche Arbeit
  \item \textbf{Modul \glqq Model Engineering\grqq\ :} Die Umsetzung des Vorhersagemodells mit bereitgestellten Trainingsdaten
\end{enumerate}

Die ganzheitliche Betrachtungsweise der Thematik ließ bereits jetzt  einen besseren Einblick in die Prozesse und Herausforderungen der Softwareprojektabwicklung mit  Machine Learning Aspekten entstehen.

Für die technische Bearbeitung der Aufgabenstellung wurden mehrere Werkzeuge verwendet. In vorangegangenen Modulen hat sich bereits die Kombination eines Git-Repositories für die LATEX-Dateien der Bearbeitung bewährt. Zusätzlich wurde probehalber das Repository auch einem Git-Projekt hinzugefügt um die Zeitschätzung des Projektes in Form von Iterations, Meilensteinen und Issues über das Git-Projekt zu gestalten. Dies erwies sich zwar als machbar, aber nicht als \glqq sauber\grqq\ , da im Repository, die tatsächlichen Issues der Portfoliophasen mit den fiktiven Issues des Projektes vermischt wurden. Eine Testversion der Software \glqq objectiF RPM\grqq\ \citep{microtool_gmbh_objectif_2024} unterstützte zwar der Entwicklung eines Verständnisses für die Herausforderungen in der Projektplanung, eine sinnvolle Herangehensweise und auch für Details der Umsetzung (wie zB Reviewmöglichkeiten für Anforderungen und \glqq Slicing\grqq\ von Use Cases), war jedoch für die tatsächliche Verwendung im Rahmen der Portfolioaufgabe zu Umfangreich und hätte den Rahmen gesprengt. Das GANTT-Diagramm wurde anschließend mit dem relativ intuitiv gehaltenem Tool \glqq YouTrack\grqq\ von JetBrains erstellt. Gerade, die Möglichkeit die Arbeitspakete hierarchisch zu strukturieren und zeitlich einfach anzupassen, machten das GANTT-Diagramm zu einem wertvollen Werkzeug, dass gute Einblicke in die Erfordernisse der Projektabwicklung bot.


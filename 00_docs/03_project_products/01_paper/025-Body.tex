\chapter{Das Vorhersagemodell}

\section{Organisiere} das Projekt mithilfe der CRISP-DM oder der MS Team Data Science Methode. Mache einen Vorschlag, wie die Ordnerstruktur eines Git-Repositories für das Projekt aufgebaut werden soll. Beachte, dass Du den finalen Code des Projekts nicht nach dieser Ordnerstruktur aufbauen musst. 

Warum für TDSP entschieden
moderner - optimiert für agile Entwicklung - gegenständliches Projekt hat ebenfalls Anfang und Ende - flexiblere Gestaltung des Arbeitsprozesses da alle großen \glqq Überschriften\grqq\ von einander abhängen (oder so...)

Der Link zu TDSP aus dem Studienscript, aber auch von den referenzierenden GitHub-Seiten aus ist auf Microsoft nicht mehr verfügbar. Das stattdessen angezeigte Dokument AI-Implementierung enthält zwar einen Verweis auf TDSP mit einem Link, dieser führt jedoch ebenfalls wieder zum selben AI-Implementierungsdokument. => Anm: schnell verändernde Branche

Schritte TDSP \citep{noauthor_microsoft-tdspdocslifecycle-detailmd_nodate}
%\begin{figure}[h]
%\centering
%\caption{Darstellung des Team Data Science Prozesses}
%\includegraphics[width=15cm]{resources/tdsp.png}\\
%Quelle: \cite{noauthor_microsoft-tdspdocslifecycle-detailmd_nodate}
%\label{fig:tdsp}
%\end{figure}

\subsection{Business Understanding} (Referenz auf vorangegangene Arbeiten)
\subsection{Data Acquisition and Understanding}
Referenz UseCase Analyse
Bestehende Daten und Hinweise auf mögliche Erweiterung
Explorative Datenanalyse


\subsection{Modeling}
Referenz UseCase Analyse
Referenz Monitoring (UseCase Analyse)




\subsection{Deployment}
Referenz Monitoring (UseCase Analyse)
Referenz Projektplanung

\subsection{Customer Acceptance}
Referenz Risiko in der Projektplanung

\subsection{Ordnerstruktur}

Git Link
Das Programm wurde vom Autor mit dem Namen \emph{\glqq functionfinder\grqq} betitelt und steht auf GitHub unter dem Link \begin{center}\url{https://github.com/GGProjects/DLMDWPMP01}\end{center} zum Download zur Verfügung.

 Die weitere Ordnerstruktur orientiert sich an einem Blog-Post von \cite{henk_griffioen_how_2017}, der, nach Meinung des Autors, ein leicht verständliches Framework für Pythonprojekte bietet. Bereitgestellt über \citep{griffioen_hgrifexample-project_2023}

Fusioniert mit \citep{noauthor_azureazure-tdsp-projecttemplate_2025} \colorbox{hellrot}{...}
Umsetzung nicht in einer Azure Umgebung, daher diesbezgl modifiziert.



In der nachfolgenden Abbildung werden die wesentlichen Ordner und Dateien der bereitgestellten Programmstruktur dargestellt. Hervorzuheben ist hierbei das Package \emph{functionfinder}, das sämtliche Pythonmodule beinhaltet, die für die Programmfunktionalität verantwortlich sind, sowie das Modul \emph{ffrunner}, über das die eigentliche Ausführung des Programmes gestartet wird.
Die übrigen Verzeichnisse des Projekts stehen für die bereitgestellten Daten (\emph{data}), die Dokumentation (\emph{docs}), die vom Programm erzeugten Ausgaben (\emph{output}), sowie für die vorgesehenen UnitTests (\emph{tests}) zur Verfügung.

\definecolor{mygray}{rgb}{0.9,0.9,0.9}
\lstset{backgroundcolor=\color{mygray},
	captionpos=b,  % sets the caption-position to bottom
	emph={functionfinder, ffrunner}, 
	emphstyle=\color{red},
	basicstyle=\small,
	frame=single,
	framextopmargin=6pt,
	framexbottommargin=6pt,
	morecomment=[l][\color{red}]{/},}

\begin{figure}[h]
\caption{Struktur des Ordneraufbaus}
\begin{tabular}{c}  % the tabular makes the listing as small as possible and centers it
\lstinputlisting[label=folder]{tree.txt}
\end{tabular}\\
\centering
Quelle: Eigene Darstellung.
\label{tab:structure}
\end{figure}

%möglicherweise Grafik hinzufügen
%Der logische Programmablauf gliedert sich, wie die angeführte Grafik zeigt, im Wesentlichen in drei Abschnitte, die auch im Hauptteil dieser Arbeit behandelt werden.




\section{Beurteile die Qualität} des zur Verfügung gestellten Datensatzes. Bereite Deine Erkenntnisse so auf und visualisiere sie so, dass Businesspartner in einer klaren und einfachen Weise die wichtigen Zusammenhänge verstehen können.

Stelle ein \section{erstes Basismodell} (ein sogenanntes Baseline-Modell) auf, sowie ein präzises Vorhersagemodell, das den Businessanforderungen genügt, nämlich die Erfolgsrate der Kreditkartenzahlungen zu erhöhen und gleichzeitig die Transaktionskosten gering zu halten.

Damit die Businesspartner Vertrauen in Dein neues \section{Modell} entwickeln, solltest du die Wichtigkeit der einzelnen erklärenden \subsection{Variablen} diskutieren und die \subsection{Modellresultate} so interpretierbar wie möglich gestalten. Außerdem ist eine detaillierte \subsection{Fehleranalyse} sehr wichtig, damit die Businesspartner auch die Schwachstellen Deiner Herangehensweise verstehen.

Im letzten Schritt des Projekts soll ein \section{Vorschlag} unterbreitet werden, wie Dein Modell in die tägliche Arbeit des Fachbereichs eingebunden werden kann, beispielsweise wie eine graphische Benutzeroberfläche (GUI) aussehen könnte. 


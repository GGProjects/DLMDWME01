\chapter{Einleitung}

\begin{addmargin}[1.27cm]{0cm}
Conversational interfaces are popular, but they are far from intelligent. Amazon' Alexa and other voice-response interfaces don't understand language. They simply launch computerized sequences in response to sonic sequences, which humans call verbal commands.
\end{addmargin}

Die Abkürzung \glqq GPT\grqq\ steht hierbei für \ac{GPT}. In einem Interview mit Melanie Subbiah, einer Autorin des ersten GPT-3 Konzepts und \ac{AI}-Ingenieur bei OpenAI, wird die Bedeutung der Abkürzung anschaulich erklärt:

Im gewählten Szenario geht es um die Herausforderung, die Planungslogik des Bereitschaftsdienstplans für den Berliner Rotkreuz-Rettungsdienst zu verbessern. Das, weiter unten beschriebene, Thema ist dem Modul \glqq Model Engineering\grqq\ des Masterstudienganges Data Science entnommen \citep{pak_aufgabenstellung_dlmdwme01_2024}, und wurde für zwei weitere Module adaptiert: 

\begin{enumerate}
  \itemsep-8pt
  \item \textbf{Data Science UseCase:} Analyse des Anwendungsfalls bis zur Präsentation für die Freigabe durch das Management.
  \item \textbf{Technische Projektplanung:} Aufbauend auf dem PitchDeck vom Projektstrukturplan über eine Kostenschätzung und das Risikomanagement bis zur Stakeholder-Analyse.
  \item \textbf{Model Engineering:} Die Umsetzung des UseCase als Projekt in einem Git-Repository mit bereitgestellten Daten bis zu einem fertiggestellten Vorhersagemodel.
\end{enumerate}

Die Motivation, dieses Thema auch im Modul \glqq Data Science UseCase\grqq\ zu bearbeiten,  besteht darin, die Problemstellung aus mehreren Blickwinkeln zu behandeln und so ein durchgängiges Verständnis für die Abwicklung realer Szenarien zu erlangen. Die ganzheitliche Betrachtungsweise der Thematik lässt einen tieferen Einblick in die Prozesse der Data Science entstehen, erhöht die Identifikation mit dem UseCase und macht somit schlussendlich auch mehr Spaß.	